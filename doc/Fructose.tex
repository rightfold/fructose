\documentclass[a4paper]{article}

\usepackage{syntax}
\setlength{\grammarindent}{8em}

\usepackage{float}
\floatstyle{boxed} 
\restylefloat{figure}

\begin{document}

\section{The Reader}

The reader is a program that parses a textual representation of a value and returns that value. The reader reads values according to the grammar shown in figure \ref{fig:readergrammar}.

\begin{figure}
\label{fig:readergrammar}
\caption{The reader grammar. Rules with capitalized names ignore surrounding whitespace, where whitespace is defined by rule $\{ \langle space \rangle \}$}
\begin{grammar}
<Value> ::= <Atom>
\alt <Boolean>
\alt <Number>
\alt <String>
\alt <List>

<Atom> ::= \{ \it{any code point in Unicode category {\bf L}, {\bf Pc}, {\bf Pd}, or {\bf S}} \}

<Boolean> ::= "#t" | "#f"

<Number> ::= <digit> \{ <digit> \}
\alt <digit> \{ <digit> \} "." <digit> \{ <digit> \}
\alt "0x" <hex-digit> \{ <hex-digit> \}

<String> ::= "\"" \{ <char> \} "\""

<char> ::= \it{any code point except "\"" and "\\"}
\alt "\\a" | "\\b" | "\\f" | "\\n" | "\\r" | "\\t" | "\\v" | "\\\\" | "\\'" | "\\\"" | "\\?"
\alt "\\u" <hex-digit> <hex-digit> <hex-digit> <hex-digit>

<List> ::= "(" \{ <Value> \} ")"

<letter> ::= "a" .. "z" | "A" .. "Z"

<digit> ::= "0" .. "9"

<hex-digit> ::= "0" .. "9" | "a" .. "f" | "A" .. "F"

<space> ::= \it{any code point with Unicode property {\bf WSpace=Y}}
\alt ","

\end{grammar}
\end{figure}

\end{document}
